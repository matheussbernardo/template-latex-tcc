\chapter{Formatos para serialização de dados}

No capítulo anterior, foram discutidas maneiras para transformar um texto 
em uma representação computacional. Para tal, foram definidas algumas abstrações
como linguagens, gramáticas e \textit{parsers}.

Como dito no Cap. \ref{sec:objective} o objetivo deste trabalho é definir
uma representação para troca de dados. Portanto é essencial discutir outras
notações existentes. Especialmente XML (Extensible Markup Language) e 
JSON (JavaScript Object Notation).

\section{XML (\textit{Extensible Markup Language})}

Um formato muito conhecido é o XML (Extensible Markup Language). 
Essa notação define os chamados documentos XML, e também define parcialmente o comportamento 
dos processadores de XML \cite{XML}. 

A origem do XML é a linguagem de marcação SGML e o XML é também portanto uma linguagem de marcação.
Isto significa que a especificação foi pensada como uma analogia ao ato de marcar um texto físico.
Como por exemplo em um processo de revisão.

A abstração principal é o chamado elemento e um documento XML contém um ou mais elementos. 
Um elemento consiste de tags de início e fim ou no caso de um elemento vazio 
consiste de um \textit{empty-element tag}. Esses elementos delimitam uma marcação
em um texto e possuem um tipo definido pelo seu nome além de permitirem a presença de atributos. 

\begin{lstlisting}[caption=Exemplo de elementos XML,label={lst:xmlelement}]
<termdef id="dt-dog" term="dog">duke</termdef>
<foo>bar</foo>
<br />
\end{lstlisting}

Entretanto, XML possui alguns problemas para representar dados. Um deles é
que não mapeia de maneira natural algumas estruturas de dados como listas. A sintaxe é
carregada e existe uma ambiguidade no uso de atributos e filhos. Além disso o texto 
é misturado com os elementos.

\section{JSON (\textit{JavaScript Object Notation})}

JSON é uma notação em formato de texto, mais nova que o XML, criada para facilitar a troca de 
dados de um servidor com o navegador web. Foi baseada na linguagem JavaScript (ECMAScript) e formalizada 
pela \citeonline{ecma404} após a proposta inicial do Douglas Crawford.
Tem o objetivo de ser fácil de ler e escrever e também ser fácil para máquinas interpretar e gerar.

Um valor JSON pode ser do tipo \textit{object}, \textit{array}, \textit{number}, \textit{string}, 
\textit{true}, \textit{false} ou \textit{null}, mapeando naturalmente para JavaScript e várias outras linguagens
dinâmicas.

\begin{lstlisting}[caption=Exemplo de JSON,label={lst:jsonobj}]
{
    "name": "Eberth",
    "age": 45,
    "friends": [
        "Richard",
        "Matthew"
    ],
    "studying": false
    "sleeping": true
    "bank_balance": null
}
\end{lstlisting}

A especificação do JSON propositalmente é simples e foi concebida explicitamente
com um subconjunto seguro implementado com o eval no JavaScript. 

O objetivo é definir apenas a sintaxe e não a semântica sobre como um valor JSON pode 
ser convertido para uma estrutura de dado de uma linguagem de programação, ainda que
muitas vezes a semântica implícita pelo eval do JavaScript seja favorecida.

Apesar de ser baseado em JavaScript as estruturas utilizadas em JSON podem ser encontradas 
nativamente nas principais linguagens de programação dinâmicas e implementações do \textit{parser}
existem em praticamente todas linguagens de programação.

A troca de dados entre sistemas usando JSON necessita de um acordo entre as partes envolvidas, e o JSON
possui limitações como ser um subconjunto do JavaScript e não possuir nenhum mecanismo de extensão.
Além disso não permite a entrada de comentários e a regra para vírgulas é restrita. O sistema de tipos 
é relativamente pobre e a sintaxe não permite o uso em modo \textit{stream} de maneira fácil.

\section{Outros Formatos}

Existem também outros formatos de texto e binários com foco na troca de dados
ou representação de arquivos de configuração. O FRED também foi influenciado 
pelos formatos \textit{Amazon Ion, TOML, edn e YAML}, estes trazem vários conceitos 
que evoluem os modelos JSON e XML.

\begin{itemize}
    \item \textbf{Amazon Ion}: Possui representação tanto em texto como binário. O formato de texto
    é compátivel com JSON porém possui um modelo de dados com mais tipos, por exemplo blobs binários e
    timestamp.
    \item \textbf{TOML}: O formato TOML tem um foco em arquivos de configuração e em ser simples, 
    entretanto o modelo de dados também é mais diverso que o do JSON com tipos para datas e binários 
    por exemplo. TOML é bastante influenciado pelo formato INI, que é amplamente utilizado, apesar de não
    possuir uma especificação formal.
    \item \textbf{edn}: Este formato possui um modelo de dados rico e suporta um 
    mecanismo de extensibilidade para indicar mais semântica aos dados, esta abstração possui 
    alguma semelhança com o XML. Também suporta \textit{streaming} facilmente diferente de JSON.
    Visto que vem da comunidade Clojure que é bastante focada em dados.
    \item \textbf{YAML}: É um formato com um modelo de dados baseado em linguagens dinâmicas
    como Perl, Ruby, Python. Tem o foco em ser fácil de ler por humanos e também possui suporte
    para \textit{streaming}.
\end{itemize}