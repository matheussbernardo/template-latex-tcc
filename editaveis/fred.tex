\chapter[FRED]{FRED}

Este capítulo discute como o projeto foi desenvolvido, especificamente
para cumprir com os objetivos definidos na Seção \ref{sec:objective}. Para esse fim
foi definido um escopo para o projeto, que consiste na especificação da linguagem FRED e 
como foram feitas as implementações e a coleta de resultados.

\section{Especificação do FRED}
\label{sec:solution}

A solução construída consistiu em implementar a linguagem FRED (Flexible REpresentation of Data).
Portanto é necessário antes definir esta linguagem, tanto do ponto de vista de sintaxe como de modelo de dados.

O FRED é uma notação criada para a troca de dados. Os
objetivos da linguagem são:

\begin{itemize}
    \item Facilidade para ler e escrever por humanos.
    \item Facilidade para construir \textit{parsers}.
    \item Suportar extensão do modelo de dados atráves de tags.
    \item Permitir a entrada de metadados.
    \item Sintaxe familiar, mas sem exigir compatibilidade com JSON.
\end{itemize}

O FRED possui um modelo de dados baseados em tipos frequentemente encontrados em linguagens
dinâmicas, semelhante ao JSON. Porém, com algumas adições e abstrações 
inspiradas em XML e outras linguagens.

Existem também características que permitem escrever FRED com facilidade. As vírgulas
em FRED são consideradas espaço em branco e isso permite por exemplo escrever objetos
e arrays com maior facilidade. Também permite dentro dos objetos escrever as chaves sem
aspas.

A linguagem FRED foi construída pensando tanto na facilidade para ler e escrever, como na
simplicidade para construir \textit{parsers}. Também possui abstrações que permitem incluir
mais significado semântico aos dados e, no futuro, incluir sistemas de validação e conversão
automática de tipo. 

O FRED foi desenvolvido pensando em alguns casos de uso. Inicialmente a linguagem
tem o uso principal para troca de dados, comunicação entre sistemas e
outras formas de representação de dados que devem ser legíveis para seres humanos
como arquivos de configuração.

Outro uso interessante é a facilidade de representar árvores sintáticas em FRED.
Pois, devido as tags \textit{union types} são facilmente representadas 
e portanto construir uma árvore sintática fica simples como no exemplo
\textbf{Mul [\$x, Add [40, 2]]}.

Essa representação possui vantagens em relação as \textit{S-expressions}, que
é uma notação utilizada para representar listas aninhadas e árvores, popular em linguagens da família 
LISP. Visto que, o papel do nó da árvore é destacado e possui um local natural para a 
inclusão de meta-informação.

\subsection{Comentários e Espaço em branco}

Comentários são permitidos em FRED e utilizam o caractere \textbf{;}. É ignorada
toda a linha depois do caractere até a quebra de linha. Em FRED a regra de 
espaço em branco é parecida com JSON exceto que o caractere vírgula também
é considerado espaço em branco.

\subsection{\textit{Null} e Booleanos}

Os valores atômicos são iguais ao JSON e possuem a mesma semântica.
Estão representados na gramática definida no Código \ref{lst:nullboolgrammar}, onde algumas produções
foram omitidas para facilitar a leitura.

\begin{lstlisting}[caption=Gramática para bools e null,label={lst:nullboolgrammar}]
atom : "null"
     | "true"
     | "false
\end{lstlisting}    

\subsection{\textit{Strings}}

\textit{Strings} são representadas com aspas duplas como em \textbf{"Valid string"} 
e aceitam qualquer caracter unicode exceto os que devem ser escapados,  
que são:

\begin{center}    
    \textbackslash b, \textbackslash t, \textbackslash n, \textbackslash f, 
    \textbackslash r, \textbackslash ", \textbackslash \textbackslash , \textbackslash xXX, 
    \textbackslash uXXXX, \textbackslash UXXXXXXXX
\end{center}

Diferentemente do JSON, existe uma notação para adicionar \textit{code points} unicode com 
dois ou oito hexadecimais. A gramática está especificada no Código \ref{lst:stringgrammar} e 
utiliza expressões regulares.

\begin{lstlisting}[caption=Gramática para strings,label={lst:stringgrammar}]
string : STRING_LITERAL
STRING_LITERAL : /"(?:[^\\"]|\\(?:[bfnrtv"\\/]
                 |x[0-9a-fA-F]{2}
                 |u[0-9a-fA-F]{4}
                 |U[0-9a-fA-F]{8}))*"/
\end{lstlisting}    

\subsection{Números}

Os números podem ser tanto decimais como inteiros e existe notação especial para
entrar números em diferentes bases (binário, octal, hexadecimal). Exemplo de inteiros 
são \textbf{42, -42, 1\_000\_000, 0xBEEF\_00E9, 0o7823, 0b1010} e exemplos de números do tipo
float são \textbf{42.12, -42.12, 4.32e-19, -2E-2}.

A gramática que especifica estes tipos está no Código \ref{lst:numbergrammar}. As definições das
expressões regulares foram omitidas.

\begin{lstlisting}[caption=Gramática para números,label={lst:numbergrammar}]
number : NUMBER_LITERAL 
    | HEX_LITERAL
    | OCT_LITERAL
    | BIN_LITERAL 
\end{lstlisting}    

\subsection{Data e Hora}

Um tipo que não existe em JSON mas que está definido na especificação do FRED é o \textit{DateTime}.
Isto é, Datas e Horas podem ser representadas em FRED. Utilizando uma notação parecida com
a RFC 3339.

Datas não associadas a horas e nem a fusos, a representação é \textbf{YYYY-MM-DD}.
Para representar Tempo sem estar associado a datas nem a fusos a representação seria \textbf{HH:MM:SS.SS}.

Datas com Horas podem ser representadas com seguinte notação \textbf{YYYY-MM-DDTHH:MM:SS.SSZ} ou
\textbf{YYYY-MM-DDTHH:MM:SS.SS+-HH:MM}, onde o fuso é opcional e \textbf{T} pode ser 
substituido por \textbf{\_}.

A gramática para estes tipos está exposta no Código \ref{lst:dategrammar}.

\begin{lstlisting}[caption=Gramática para datas e horas,label={lst:dategrammar}]
atom : date_time 
     | TIME_FORMAT

date_time : date
          | TIME_FORMAT

date : DATE_FORMAT [ ("_" | "T") TIME_FORMAT [TIME_OFFSET]]
\end{lstlisting}  

\subsection{\textit{Blobs} Binários}

Outro tipo de dado que não tem especificação em JSON porém tem em FRED são os blobs binários. Eles
representam dados binários codificados de alguma forma. A notação é similar a uma string precedida
por um , como no Código \ref{lst:blobgrammar}.

\begin{lstlisting}[caption=Gramática para blobs,label={lst:blobgrammar}]
atom : blob

blob : "#" BLOB_LITERAL
\end{lstlisting}

\subsection{\textit{Symbols}}

\textit{Symbols} representam semanticamente constantes ou variáveis e estão 
presentes nos modelos de dados das linguagens do tipo Lisp e em Ruby. 
Eles podem ser representados em FRED usando o caracter \textbf{\$}. Por Exemplo, \textbf{\$var1}.

A gramática que especifica esse tipo está no Código \ref{lst:symbolgrammar}.

\begin{lstlisting}[float,floatplacement=H,caption=Gramática para symbols,label={lst:symbolgrammar}]
atom : symbol

symbol : "$" name
\end{lstlisting}

\subsection{Listas e Arrays}

A linguagem FRED também permite a representação de listas e/ou arrays, utilizando uma notação parecida com JSON.
Por exemplo, \textbf{[1 2 3]}. Vale notar que o separador é pelo menos um espaço e que vírgulas em FRED
são consideradas espaço em branco. Logo, listas também podem ser escritas desta forma \textbf{[1, 2, 3]} ou 
até mesmo de maneira misturada como em \textbf{[1, 2 3]}.

A gramática que define esse tipo está no Código \ref{lst:listgrammar}.

\begin{lstlisting}[caption=Gramática para listas,label={lst:listgrammar}]
atom : array

array : "[" value* "]"
\end{lstlisting}

\subsection{Objetos}

Também é possível representar objetos em FRED. A sintaxe é muito parecida com JSON. 
Exemplo de dicionário em FRED pode ser visto no Código \ref{lst:fredobj}. Basicamente é um conjunto de pares 
chave e valor separados por pelo menos um espaço.

\begin{lstlisting}[caption=Exemplo de dicionário em FRED chaves com espaços são escapados com backticks,label={lst:fredobj}]
{
    foo : "bar"
    `test foo` : "bar"
}
\end{lstlisting}

A gramática que define o tipo objeto está no Código \ref{lst:objgrammar}.

\begin{lstlisting}[caption=Gramática para objetos,label={lst:objgrammar}]
atom : object

object : "{" pair* "}"

pair : name ":" value

name : VARIABLE
     | QUOTED_VARIABLE
\end{lstlisting}
    
\subsection{Tags e Metadados}

Existem duas abstrações importantes em FRED que não existem em JSON. São as \textit{tags} e a possibilidade de entrada 
de metadados.

\begin{lstlisting}[caption=Exemplo de dicionário com Tag em FRED,label={lst:fredtagobj}]
person {
    name: "eric"
    age: 25
}
\end{lstlisting}
    
As \textit{tags} são uma maneira de indicar significados específicos para os valores em FRED. Isto permite
mecanismos de extensão do modelo de dados base. No Código \ref{lst:fredtagobj} 
temos um objeto com tag em FRED.

Valores com tag em FRED podem ser associados a metadados. A sintaxe consiste em entrar 
atributos dentro de parentêses. Um exemplo de um valor FRED com tag e metadados 
é \textbf{phone (country="Brazil") "32131123"}.

Também é possível representar tags e metadados sem estar associado a nenhum elemento. 
Por exemplo: \textbf{(tag attr=1)}. Isto é semanticamente igual a uma tag associada ao valor null
e corresponde às \textit{empty tags} do XML.

A gramática que define a sintaxe das Tags está no Código \ref{lst:taggrammar}.

\begin{lstlisting}[caption=Gramática para tags e metadados,label={lst:taggrammar}]
value : tagged
      | atom

tagged : name [attrs] atom
       | "(" name attr* ")" 

attrs : "(" attr* ")"

attr : name "=" atom
\end{lstlisting}

\subsection{Suporte para \textit{Streaming} e \textit{Append Only}}

FRED possui suporte para \textit{streaming}, isto é funciona em 
arquivos grandes que não são lidos inteiramente em memória mas sim como um fluxo de dados. 
No Código \ref{lst:fredstream} está um exemplo de documento FRED para \textit{streaming}. Isto permite
utilizar documentos FRED em modo \textit{append only}, o que é muito útil
para \textit{logs} e em várias aplicações onde os dados são tratados como imutáveis.

\begin{lstlisting}[float,floatplacement=H,caption=Exemplo de documento com streaming em FRED,label={lst:fredstream}]
---
person "Jhon"
---
person "Mary"
---
person "James"
---
\end{lstlisting}

A gramática para suportar \textit{streaming} está no Código \ref{lst:streamgrammar}.

\begin{lstlisting}[caption=Gramática para streaming,label={lst:streamgrammar}]
document : stream
         | value

stream : "---" (value  "---")*
\end{lstlisting}

\subsection{Gramática completa e especificação}

A gramática do FRED foi especificada formalmente no Apêndice \ref{sec:fredgrammar}.
Ela foi desenvolvida de forma a ser fácil de criar \textit{parsers} usando técnicas 
simples como o LL(1) e descida recursiva. Entretando, isto não significa que as implementações
devam utilizar estas técnicas.

No Apêndice \ref{sec:fredtrain} a gramática especificada pode ser visualizada
por meio de diagramas de sintaxe.

Toda essa especificação está divulgada de maneira aberta no 
endereço \url{https://github.com/fred-format/fred}.
 
\section{Atividades do Projeto de Implementação}

Com a especificação formal da linguagem FRED, é necessário 
descrever o escopo do projeto de implementação. As implementações foram feitas 
em duas linguagens: Haskell e JavaScript. Também foi discutido
a linguagem FRED e como ela se relaciona com outros formatos.

As atividades que foram feitas nesse projeto são:

\begin{itemize}
    \item Implementação do FRED em Haskell
    \begin{itemize}
        \item Estudo de arquitetura e tecnologias.
        \item Implementar a linguagem em Haskell.
        \item Documentar a implementação.
        \item Distribuir a implementação no Hackage.
    \end{itemize}
    \item Implementação do FRED em JavaScript
    \begin{itemize}
        \item Estudo de arquitetura e tecnologias.
        \item Implementar a linguagem em JavaScript.
        \item Documentar a implementação.
        \item Distribuir a implementação no NPM.
    \end{itemize}
\end{itemize}

A implementação da linguagem foi desenvolvida acompanhada de uma suíte de testes 
para garantir que as implementações de FRED estão conforme a especificação.
As atividades que foram feitas nessa etapa são:

\begin{itemize}
    \item Estudar como construir testes agnósticos à linguagens de programação.
    \item Construir uma suíte de testes capaz de testar implementações de FRED.
    \item Documentar a suíte de testes.
\end{itemize}

Por fim foi feito uma análise simples da linguagem FRED baseada em algumas métricas
de memória. Essa análise consiste de uma comparação com linguagens semelhantes a 
partir de métricas definidas.

As  atividades desenvolvidas nessa etapa desenvolvidas foram coletar dados e 
comparar FRED com linguagens semelhantes.