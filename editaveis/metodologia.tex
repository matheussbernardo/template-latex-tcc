\chapter[Metodologia]{Metodologia}

\section{Solução proposta}

Para cumprir com os objetivos definidos na Seção \ref{sec:objective}. É necessário definir
o escopo deste projeto. A solução a ser construída consiste em implementar a linguagem FRED.
Portanto é necessário antes definir esta linguagem.

O FRED (Flexible REpresentation of Data) é uma notação criada para a troca de dados. Os
objetivos dela são:

\begin{itemize}
    \item Facilidade para ler e escrever por humanos
    \item Facilidade para construir \textit{parsers}
    \item Suportar extensão atráves de tags
    \item Permitir a entrada de metadados
\end{itemize}

O FRED consiste dos tipos tradicionais encontrados em JSON, porém com algumas adições e outras abstrações
inspiradas em XML e outras notações.

Existe o valor representando \textit{null} e como esperado a notação é  \textbf{null}. Booleanos
também são representados como esperado \textbf{true} e \textbf{false}.

Strings são representadas com aspas duplas e aceitam qualquer caracter unicode 
exceto os que devem ser \textit{escaped}. Está representado uma string válida em \ref{lst:fredstring} 
e os caracteres que precisam de escape em \ref{lst:fredescape}.

\begin{lstlisting}[caption=Exemplo de String em FRED,label={lst:fredstring}]
"String that is valid"
\end{lstlisting}

\begin{lstlisting}[caption=Caracteres que devem ser escaped,label={lst:fredescape}]
\b, \t, \n, \f, \r, \", \\, \xXX, \uXXXX, \UXXXXXXXX
\end{lstlisting}

Os números podem ser tanto \textit{floats} como inteiros e existe notação especial para
entrar números em diferentes bases (binário, octal, hexadecimal). Exemplo de inteiros 
são \textbf{42 -42 1\_000\_000 0xBEEF\_00E9 0o7823 0b1010}. E exemplos de números do tipo
float são \textbf{42.12 -42.12 4.32e-19 -2E-2}.

Um tipo que não existe em JSON mas está definido na spec do FRED são os DateTime.
Datas e Horas podem ser representadas em FRED. Utilizando uma notação parecida com
a RFC 3339.

Para Datas isoladas não associadas a fusos e a horas, a representação é \textbf{YYYY-MM-DD}.

Para representar Tempo sem estar associado a nenhum fuso a representação é \textbf{HH:MM:SS.SS}.

Datas com Horas podem ser representadas sem estar associado a nenhum fuso horário. 
A notação é \textbf{YYYY-MM-DDTHH:MM:SS.SS} e o T pode ser substituido por \textbf{\_}.

Por fim pode-se representar Datas com Horas e fusos. Utilizando a notação anterior e adicionando
ou um \textbf{Z} para representar UTC ou \textbf{+-HH:MM} representando um fuso.

Outro tipo de dado que não tem especificação em JSON porém tem em FRED são blobs. Eles
representam dados binários codificados de alguma forma. A notação é \textbf{\#"rawdata"}.

\textit{Symbols} são muito conhecidos em Ruby. Eles podem ser representados em FRED usando o caracter
\textbf{\$}. Exemplo \textbf{\$var1}.

A linguagem FRED também permite a representação de Listas. Utilizando a notação do JSON.
\textbf{[1 2 3]}, vale notar que o separador é pelo menos um espaço e que vírgulas em FRED
são consideradas espaço em branco. Logo listas também podem ser escritas desta forma \textbf{[1, 2, 3]}.

Também é possível representar Dicionários em FRED. A sintaxe é muito parecida com JSON. Porém
possui algumas diferenças para facilitar. Exemplo de dicionário em FRED em \ref{lst:fredobj}.
Basicamente é um conjunto de pares chave/valor separados por pelo menos um espaço.

\begin{lstlisting}[caption=Exemplo de dicionário em FRED,label={lst:fredobj}]
{
    foo : "bar"
    `test foo` : "bar"
}
\end{lstlisting}

Existem duas abstrações em FRED que não existem em JSON. São as \textit{tags} e a possibilidade de entrada 
de metadados.

\begin{lstlisting}[caption=Exemplo de dicionário com Tag em FRED,label={lst:fredtagobj}]
person {
    name: "eric"
    age: 25
}
\end{lstlisting}
    
As \textit{tags} são uma maneira de indicar mais significado semântico para os elementos em FRED. E pode
em trabalhos futuros ter mecanismos de verificação e extensão. No exemplo \ref{lst:fredtagobj} 
temos um dicionário com tag em FRED.

Quando se tem elementos com tag em FRED é possível entrar metadados sobre este elemento utilizando uma notação
especial. A sintaxe consiste de entrar atributos dentro de parentêses. Exemplo de elementos com 
tag e metadados \textbf{phone (country="Brazil") "32131123"}.

Também é possível representar tags e metadados sem estar associado a nenhum elemento. 
Exemplo: \textbf{(tag attr=1)}.

FRED tem suporte para \textit{streaming}. Em \ref{lst:fredstream} está
um exemplo de documento FRED para \textit{streaming}.

\begin{lstlisting}[caption=Exemplo de documento com streaming em FRED,label={lst:fredstream}]
---
person "Jhon"
---
person "Mary"
---
person "James"
---
\end{lstlisting}

A gramática do FRED foi especificada formalmente no Apêndice \ref{sec:fredgrammar}.
Ela foi desenvolvida de forma a ser fácil de criar \textit{parsers} LL(1).
 
\section{Atividades do Projeto}

Dado a especificação formal da linguagem FRED, esta será implementada em Haskell e JavaScript.
As atividades a serem feitas nessa etapa são:

\begin{itemize}
    \item Implementação do FRED em Haskell
    \begin{itemize}
        \item Escolher biblioteca para construir o \textit{parser}.
        \item Construir o \textit{parser}.
        \item Documentar o \textit{parser}.
        \item Distribuir o \textit{parser} no Hackage.
    \end{itemize}
    \item Implementação do FRED em JavaScript
    \begin{itemize}
        \item Escolher biblioteca para construir \textit{parser}.
        \item Construir o \textit{parser}.
        \item Documentar o \textit{parser}.
        \item Distribuir o \textit{parser} no npm.
    \end{itemize}
\end{itemize}

Com a implementação da linguagem será desenvolda uma suíte de testes para garantir 
que as implementações de FRED estão conforme a especificação. As atividades a serem
feitas nessa etapa são:

\begin{itemize}
    \item Suíte de Testes
    \begin{itemize}
        \item Estudar como construir testes para \textit{parsers}.
        \item Construir biblioteca capaz de testar implementações de FRED.
        \item Documentar biblioteca.
    \end{itemize}
\end{itemize}

Por fim será feito uma análise da linguagem FRED. Essa análise consiste 
em uma discussão sobre a linguagem e também será feito alguma comparação
com linguagens semelhantes a partir de métricas a serem definidas.

As  atividades a serem desenvolvidas nessa etapa são:

\begin{itemize}
    \item Discutir as características diferenciais da linguagem FRED.
    \item Coletar dados e Comparar FRED com linguagens semelhantes.
\end{itemize}
