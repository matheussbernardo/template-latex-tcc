\begin{resumo}

Este projeto propõe a especificação e a implementação de um formato para
troca e representação de dados, este formato é o FRED. Essa notação foi 
especificada com uma sintaxe inspirada em JSON porém evoluí o modelo de 
dados com mais tipos, por exemplo, dados representado data, hora e blobs 
binários. Além disso, a especificação prevê um mecanismo de extensão 
para este modelo de dados, inspirado em XML e também com influência de 
outros formatos correlatos. Outro mecanismo diferencial em relação a JSON 
é uma notação específica para a entrada de metadados. Já a implementação do 
formato FRED foi realizada no projeto em duas linguagens Haskell e JavaScript 
e também foi desenvolvida uma suíte de testes para validar a implementação 
de acordo com a especificação. Por fim foi realizada uma comparação simples 
entre FRED e outros formatos e percebe-se portanto a necessidade de mais 
estudos comparativos em vários casos de uso.

 \vspace{\onelineskip}
    
 \noindent
 \textbf{Palavras-chaves}: representação de dados. troca de dados. json. xml.
\end{resumo}
