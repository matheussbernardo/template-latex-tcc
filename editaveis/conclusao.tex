\chapter[Conclusão]{Conclusão}

O projeto consistiu em especificar e implementar uma linguagem para troca de dados.
A especificação foi construída com o objetivo de facilitar as implementações.
Para tal, a gramática permite desenvolver parsers com técnicas simples como 
LL(1) e descida recursiva.

A primeira implementação foi feita em Haskell e utiliza a técnica
de \textit{parsers combinators}. Neste analisador a separação entre léxico, sintático
e semântico não ficou clara. Entretanto o código foi modularizado por meio
das funções e módulos que estão associados a partes da linguagem.

Já o segundo analisador foi escrito em JavaScript. Esta implementação
utilizou uma biblioteca que constrói parsers LL(k). A arquitetura foi dividida 
de maneira clara entre o analisador léxico, sintático e semântico, pois
a biblioteca também separa essas responsabilidades.

Com as implementações desenvolvidas foi necessário uma forma de verificar se estas
condizem com a especificação. Para tanto, uma suíte de testes agnóstica a linguagem
de implementação foi desenvolvida. Ela consiste de dois conjuntos de testes
que verificam documentos FRED válidos e inválidos de acordo com a especificação.

As implementações portanto utilizam essa suíte e verificam se estão de 
acordo com a especificação.

Durante o desenvolvimento do projeto foi realizado uma análise a 
respeito da linguagem FRED e suas características. Também foi realizado
uma comparação simples com outros formatos correlatos. Entretanto,
não é posível afirmar que as características diferenciais melhoram
a usabilidade ou a legibilidade. Necessitando desenvolver melhor o
ecossistema e realizar estudos comparativos mais formais.

A linguagem FRED foi especificada e implementada em dois ambientes,
entretanto é necessário mais estudos sobre as vantagens e desvantagens
de utilizá-la e também desenvolver projetos utilizando FRED.

\section{Trabalhos Futuros}

Como sequência a este projeto, percebe-se a necessidade de
um estudo comparativo entre as linguagens correlatas, especialmente JSON e XML.

Não era parte do projeto fazer esta comparação de maneira formal. Entretanto,
este estudo pode fornecer informações úteis para futuras evoluções da linguagem
e do seu ecossistema.

Também para melhorar o ecossistema do FRED, é necessário estudos
sobre verificação e validação. E possivelmente definir uma especificação 
de um formato para \textit{schemas}.

Outro trabalho futuro é experimentar a linguagem em diferentes casos de uso
com o objetivo de entender onde existem vantagens e desvantagens no seu uso.
