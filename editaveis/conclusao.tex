\chapter[Conclusão]{Conclusão}

O projeto consistiu em especificar e implementar uma linguagem para troca de dados, chamada FRED.
A especificação foi construída com o objetivo de facilitar a implementação e portanto
utiliza recursos simples.

A primeira implementação foi feita em Haskell e utiliza a técnica
de \textit{parsers combinators}. Neste analisador, a separação entre léxico, sintático
e semântico não ficou clara. Algo que é comum neste tipo de técnica. 
Entretanto, o código foi modularizado por meio 
das funções e módulos que estão associados a partes da linguagem.

O segundo analisador foi escrito em JavaScript. Esta implementação
utilizou uma biblioteca que constrói parsers LL(k). A arquitetura foi dividida 
de maneira clara entre o analisador léxico, sintático e semântico, pois
a biblioteca também separa essas responsabilidades.

Com as implementações desenvolvidas foi necessário uma forma de verificar se estas
condizem com a especificação. Para tanto, uma suíte de testes agnóstica a linguagem
de implementação foi desenvolvido. Ela consiste de dois conjuntos de testes
que verificam documentos FRED válidos e inválidos de acordo com a especificação.
As implementações portanto utilizam essa suíte e verificam se estão de 
acordo com a especificação.

Durante o desenvolvimento do projeto, realizamos uma análise a 
respeito da linguagem FRED e suas características. Entretanto,
não é posível afirmar que as características diferenciais melhoram
a usabilidade ou a legibilidade. Para tal, é necessário desenvolver melhor o
ecossistema e realizar estudos comparativos mais formais.

Também fizemos uma comparação simples com outros formatos correlatos, 
a métrica coletada foi o tamanho dos arquivos, percebe-se que os 
arquivos FRED são menores nos casos de uso explorados. Porém estudos mais
formais e amplos são necessários visto que não é parte do escopo do projeto
um estudo comparativo formal.

O formato FRED foi especificado e implementado em apenas duas linguagens de programação,
portanto são necessários mais estudos sobre as vantagens e desvantagens
de utilizar este format em outras linguagens. Para tal, é preciso mais realizar mais projetos 
utilizando FRED e portanto implementar FRED em outras linguagens é essencial.

Faltam também estudos relacionados a performance de \textit{parsing}. Já que, não foi 
realizado pois não existem ainda \textit{parsers} de FRED implementados manualmente e calibrados 
para performance máxima. Isto distorceria os resultados já que as alternativas para XML e JSON
existem e são extremamente maduras.

\section{Trabalhos Futuros}

Como sequência a este projeto, percebe-se a necessidade de
um estudo comparativo entre as linguagens correlatas, especialmente JSON e XML.

Não era escopo do projeto fazer esta comparação. Entretanto,
este estudo pode fornecer informações úteis para futuras evoluções da linguagem
e do seu ecossistema.

Também para melhorar o formato FRED, é necessário estudos sobre verificação e 
validação de modelo de dados e com isso definir uma especificação 
de um formato para \textit{schemas}.

É necessário um estudo de performance de \textit{parsing}, porém primeiro
é necessário implementar JSON e XML numa biblioteca de \textit{parsing} comum,
como a utilizada nas implementações deste trabalho, ou implementar FRED 
de forma otimizada em uma linguagem como C.

Em relação a performance vale a pena mencionar que JSON é compatível com LL(1) a 
nível de caractere individual e isto permite implementações mais eficientes pois não 
é necessário fazer a análise léxica separadamente. Já FRED provavelmente não é devido 
a ambiguidades, por exemplo com o ínicio de null e uma tag que começa com n.

Implementar o formato FRED em outras linguagens de programação
é extremamente necessário para possibilitar mais estudos e experimentos
em diferentes casos de uso. 

Uma dessas implementações a serem feitas deve utilizar as 
técnicas de descida recursiva e  LL(1) para garantir 
que a gramática é de fato compatível com essas premissa.

Também é necessário estabilizar a sintaxe, pois a partir da opinião
da comunidade e de trabalhos futuros, possíveis melhorias e mudanças serão
sugeridas.
