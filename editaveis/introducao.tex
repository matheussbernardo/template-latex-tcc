\chapter[Introdução]{Introdução}

Na computação os formatos utilizados para representar dados
possuem muita importância pois a maneira em que eles são organizados
influem tanto na leitura e escrita por humanos como também 
na interpretação pelos computadores.

Com o advento da internet a comunicação entre sistemas aumentou.
Por isso foi necessário a criação de formatos padronizados para representar
os dados comunicados. Dentre estes destaca-se XML e JSON.

O XML foi concebido a partir de uma linguagem de marcação chamada SGML.
Um de seus objetivos é ser de fácil utilização na internet \cite{XML}.

O formato JSON foi criado com base na representação de objetos 
da linguagem JavaScript, possui uma sintaxe simples e tem como
objetivo ser fácil tanto para humanos ler e escrever como
para máquinas interpretar \cite{ecma404}.

Nesse projeto é especificado o FRED (\textit{Flexible REpresentation of Data}), 
um formato para representação e troca de dados. Este formato possui uma sintaxe
inspirada no JSON porém incrementa o modelo de dados utilizando algumas 
abstrações que lembram o XML.

Além disso o FRED possui um modelo de dados um pouco mais complexo que JSON,
permitindo por exemplo dados sobre data e hora, além disso mais tipos estão
definidos na especificação.

Este trabalho propõe especificar e implementar a linguagem FRED, para tal
um estudo sobre linguagens formais e compiladores foi feito com o objetivo
de fornecer a base teórica necessária.

\section{Objetivos}
\label{sec:objective}

\subsection{Objetivo Geral}

O objetivo principal deste trabalho é a construção de uma linguagem para
representação e troca de dados, com base na sintaxe de JSON e
utilizando de alguns conceitos do XML para expandir as abstrações 
desta nova linguagem.

\subsection{Objetivos Específicos}

\begin{itemize}
    \item Especificar a linguagem de maneira formal.
    \item Implementar a linguagem com base na especificação.
    \item Analisar e discutir a linguagem implementada.
\end{itemize}

\section{Organização do trabalho}

Este trabalho está organizado em três partes principais:

\begin{itemize}
    \item \textbf{Referencial Teórico:} Esta seção trata da fundamentação teórica
    utilizada no trabalho e tem o objetivo de fornecer a base para o desenvolvimento 
    do projeto.
    \item \textbf{Metodologia:} Esta seção define como será feito o trabalho e também
    o escopo do projeto.
    \item \textbf{Resultados:} Esta seção aborda os resultados que foram alcançados no decorrer
    do projeto.
\end{itemize}