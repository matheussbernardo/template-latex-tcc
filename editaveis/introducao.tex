\chapter[Introdução]{Introdução}

\section{Objetivos}
\label{sec:objective}

\subsection{Objetivo Geral}

O objetivo principal deste trabalho é construir uma linguagem para troca de dados
com foco em prover significado semântico aos dados.

\subsection{Objetivos Específicos}

\begin{itemize}
    \item Especificar a linguagem de maneira formal.
    \item Implementar a linguagem com base na especificação.
    \item Analisar a linguagem implementada.
\end{itemize}

\section{Organização do trabalho}

Este trabalho está organizado em três partes principais:

\begin{itemize}
    \item \textbf{Referencial Teórico:} Esta seção trata da fundamentação teórica
    utilizada no trabalho e tem o objetivo de fornecer a base para entender o projeto.
    \item \textbf{Metodologia:} Esta seção define como será feito o trabalho e também
    o escopo do projeto.
    \item \textbf{Resultados:} Esta seção aborda os resultados que foram alcançados no decorrer
    do projeto.
\end{itemize}