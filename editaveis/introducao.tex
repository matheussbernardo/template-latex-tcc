\chapter[Introdução]{Introdução}

Formatos utilizados para representar dados possuem muita importância 
em vários sistemas computacionais, pois a maneira em que eles são organizados
e o poder de representação influem tanto na leitura e escrita por humanos como também 
na interpretação pelos computadores.

Com o advento da internet, a comunicação entre sistemas aumentou.
Por isso foi necessário a criação de formatos padronizados para representar
os dados utilizados na comunicação de sistemas que frequentemente adotam 
tecnologias diferentes e diferentes linguagens de programação. 
Os dois formatos mais utilizados hoje em dia são XML e JSON.

O XML foi concebido a partir de uma linguagem de marcação, SGML.
Um de seus objetivos é ser de fácil utilização na internet e representam 
documentos utilizando a metáfora de marcações de texto \cite{XML}.

O formato JSON foi criado com base na representação de objetos 
da linguagem JavaScript e é um subconjunto da linguagem, que
possui uma sintaxe simples e tem como objetivo ser fácil tanto para 
humanos ler e escrever como para máquinas interpretar \cite{ecma404}.

Nesse projeto especificamos o FRED (\textit{Flexible REpresentation of Data}), 
um formato para representação e troca de dados. Este formato possui uma sintaxe
inspirada no JSON porém incrementa o modelo de dados utilizando algumas 
abstrações encontradas no XML.

Além disso, o FRED possui um modelo de dados um pouco mais complexo que JSON,
permitindo por exemplo dados sobre data e hora, entre outros
definidos na especificação.

Visto que, XML representa dados com o foco em marcação e JSON 
possui um modelo de dados muito simples. O FRED possui uma sintaxe
inspirada em JSON porém aumenta o modelo
de dados e também permite extender este modelo mapeando naturalmente os 
dados representados para como eles são usados 
frequentemente. Isto é, facilitar a representação de dados por exemplo
quando usados como classes, enums, entre outros.



\section{Objetivos}
\label{sec:objective}

\subsection{Objetivo Geral}

O objetivo principal deste trabalho é a construção de uma linguagem para
representação e troca de dados, com base na sintaxe de JSON e
utilizando de alguns conceitos do XML para expandir as abstrações 
desta nova linguagem.

\subsection{Objetivos Específicos}

\begin{itemize}
    \item Especificar a linguagem de maneira formal.
    \item Implementar a linguagem com base na especificação.
    \item Implementar o \textit{parser} em mais de uma linguagem de programação.
    \item Analisar e discutir a linguagem implementada.
\end{itemize}

\section{Organização do trabalho}

Este trabalho está organizado em três partes principais:

\begin{description}
    \item[Referencial Teórico] Esta seção trata da fundamentação teórica
    utilizada no trabalho e tem o objetivo de fornecer a base para o desenvolvimento 
    do projeto.
    \item[Metodologia] Esta seção define como será feito o trabalho e também
    o escopo do projeto.
    \item[Resultados] Esta seção aborda os resultados que foram alcançados no decorrer
    do projeto.
\end{description}